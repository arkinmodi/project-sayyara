\documentclass[12pt, titlepage]{article}

\usepackage{fullpage}
\usepackage[round]{natbib}
\usepackage{multirow}
\usepackage{booktabs}
\usepackage{tabularx}
\usepackage{graphicx}
\usepackage{float}
\usepackage{hyperref}
\usepackage[nottoc,numbib]{tocbibind}
\usepackage{enumitem}
\usepackage{longtable}

\hypersetup{
	colorlinks,
	citecolor=blue,
	filecolor=black,
	linkcolor=red,
	urlcolor=blue
}

\input{../../Comments}
%% Common Parts

\newcommand{\progname}{Software Engineering}
\newcommand{\authname}{Team 3, Tiny Coders
	\\ Arkin Modi
	\\ Joy Xiao
	\\ Leon So
	\\ Timothy Choy} % AUTHOR NAMES

\usepackage{hyperref}
\hypersetup{colorlinks=true, linkcolor=blue, citecolor=blue, filecolor=blue,
	urlcolor=blue, unicode=false}
\urlstyle{same}

\usepackage{parskip}
\usepackage{geometry}
\geometry{a4paper, portrait, margin=1in}


\begin{document}

\title{Module Guide for \progname{}}
\author{\authname}
\date{\today}

\maketitle

\pagenumbering{roman}

\section{Revision History}

\begin{table}[hp]
	\caption{Revision History} \label{TblRevisionHistory}
	\begin{tabularx}{\textwidth}{llX}
		\toprule
		\textbf{Date}     & \textbf{Developer(s)} & \textbf{Change}                                    \\
		\midrule
		December 28, 2022 & Arkin Modi            & Create Revision History                            \\
		January 3, 2023   & Arkin Modi            & Add Module Hierarchy                               \\
		January 8, 2023   & Arkin Modi            & Add Use Hierarchy Between Modules Diagram          \\
		January 10, 2023  & Arkin Modi            & Add Module Decompositions                          \\
		January 12, 2023  & Leon So               & Anticipated Changes \& Unlikely Changes            \\
		January 12, 2023  & Arkin Modi            & Add Traceability Matrices                          \\
		January 15, 2023  & Arkin Modi            & Add ``Type of Module" to Behaviour-Hiding Modules  \\
		January 15, 2023  & Arkin Modi            & Merge Shop Profile \& Shop Lookup Module           \\
		March 5, 2023     & Leon So               & Update Traceability Matrix for Shop Profile Module \\
		\bottomrule
	\end{tabularx}
\end{table}

\newpage

\section{Reference Material}

This section records information for easy reference.

\subsection{Abbreviations and Acronyms}

\renewcommand{\arraystretch}{1.2}
\begin{tabular}{l l}
	\toprule
	\textbf{symbol} & \textbf{description}                \\
	\midrule
	AC              & Anticipated Change                  \\
	DAG             & Directed Acyclic Graph              \\
	M               & Module                              \\
	MG              & Module Guide                        \\
	ORM             & Object-relational Mapping           \\
	OS              & Operating System                    \\
	R               & Requirement                         \\
	SC              & Scientific Computing                \\
	SRS             & Software Requirements Specification \\
	\progname       & The name of the program being built \\
	UC              & Unlikely Change                     \\
	CRUD            & Create, Read, Update, and Delete    \\
	\bottomrule
\end{tabular}

\newpage

\tableofcontents

\newpage

\listoftables

\listoffigures

\newpage

\pagenumbering{arabic}

\section{Introduction}

Decomposing a system into modules is a commonly accepted approach to developing software. A module
is a work assignment for a programmer or programming team~\citep{ParnasEtAl1984}. We advocate a
decomposition based on the principle of information hiding~\citep{Parnas1972a}. This principle
supports design for change, because the ``secrets'' that each module hides represent likely future
changes. Design for change is valuable in SC, where modifications are frequent, especially during
initial development as the solution space is explored.

Our design follows the rules layed out by \citet{ParnasEtAl1984}, as follows:
\begin{itemize}
	\item System details that are likely to change independently should be the secrets of separate modules.
	\item Each data structure is implemented in only one module.
	\item Any other program that requires information stored in a module's data structures must obtain it by
	      calling access programs belonging to that module.
\end{itemize}

After completing the first stage of the design, the Software Requirements Specification (SRS), the
Module Guide (MG) is developed~\citep{ParnasEtAl1984}. The MG specifies the modular structure of
the system and is intended to allow both designers and maintainers to easily identify the parts of
the software. The potential readers of this document are as follows:

\begin{itemize}
	\item New project members: This document can be a guide for a new project member to easily understand the
	      overall structure and quickly find the relevant modules they are searching for.
	\item Maintainers: The hierarchical structure of the module guide improves the maintainers' understanding
	      when they need to make changes to the system. It is important for a maintainer to update the
	      relevant sections of the document after changes have been made.
	\item Designers: Once the module guide has been written, it can be used to check for consistency,
	      feasibility, and flexibility. Designers can verify the system in various ways, such as consistency
	      among modules, feasibility of the decomposition, and flexibility of the design.
\end{itemize}

The rest of the document is organized as follows. Section \ref{SecChange} lists the anticipated and
unlikely changes of the software requirements. Section \ref{SecMH} summarizes the module
decomposition that was constructed according to the likely changes. Section \ref{SecConnection}
specifies the connections between the software requirements and the modules. Section \ref{SecMD}
gives a detailed description of the modules. Section \ref{SecTM} includes two traceability
matrices. One checks the completeness of the design against the requirements provided in the SRS.
The other shows the relation between anticipated changes and the modules. Section \ref{SecUse}
describes the use relation between modules.

\section{Anticipated and Unlikely Changes} \label{SecChange}

This section lists possible changes to the system. According to the likeliness of the change, the
possible changes are classified into two categories. Anticipated changes are listed in Section
\ref{SecAchange}, and unlikely changes are listed in Section \ref{SecUchange}.

\subsection{Anticipated Changes} \label{SecAchange}

Anticipated changes are the source of the information that is to be hidden inside the modules.
Ideally, changing one of the anticipated changes will only require changing the one module that
hides the associated decision. The approach adapted here is called design for change.

\begin{enumerate}[label=\textbf{AC\arabic*:},ref=AC\arabic*]
	\item \label{acHardware} The specific hardware on which the software is running.
	\item \label{acInitialInput} The format of the initial input data.
	\item \label{acServerSideResponses} The format of server-side responses
	\item \label{acStyling} The styling of UI forms and components
	\item \label{acFormatting} The formatting of UI forms and components
\end{enumerate}

\subsection{Unlikely Changes} \label{SecUchange}

The module design should be as general as possible. However, a general system is more complex.
Sometimes this complexity is not necessary. Fixing some design decisions at the system architecture
stage can simplify the software design. If these decision should later need to be changed, then
many parts of the design will potentially need to be modified. Hence, it is not intended that these
decisions will be changed.

\begin{enumerate}[label=\textbf{UC\arabic*:},ref=UC\arabic*]
	\item \label{ucNextjs} The application will be built-upon the Next.js framework
	\item \label{ucPrisma} The application will use Prisma for ORM
	\item \label{ucInput} There will always be a source of input data external to the system
\end{enumerate}

\section{Module Hierarchy} \label{SecMH}

This section provides an overview of the module design. Modules are summarized in a hierarchy
decomposed by secrets in Table \ref{TblMH}. The modules listed below, which are leaves in the
hierarchy tree, are the modules that will actually be implemented.

\begin{enumerate}[label=\textbf{M\arabic*:},ref=M\arabic*]
	\item \label{mHH} Hardware-Hiding Module
	\item \label{mDBDriver} Database Driver Module
	\item \label{mUsers} Users Module
	\item \label{mQuotes} Quotes Module
	\item \label{mAppointments} Appointments Module
	\item \label{mWorkOrders} Work Orders Module
	\item \label{mEmployeeManagement} Employee Management Module
	\item \label{mServices} Services Module
	\item \label{mShop} Shop Module
\end{enumerate}

Note that \ref{mHH} is a commonly used module and is already implemented by the operating system.
It will not be reimplemented.

\begin{table}[H]
	\centering
	\begin{tabular}{p{0.3\textwidth} p{0.6\textwidth}}
		\toprule
		\textbf{Level 1}                                      & \textbf{Level 2}           \\
		\midrule

		Hardware-Hiding Module                                & ~                          \\

		\midrule

		\multirow{7}{0.3\textwidth}{Behaviour-Hiding Module}  & Users Module               \\
		                                                      & Quotes Module              \\
		                                                      & Appointments Module        \\
		                                                      & Work Orders Module         \\
		                                                      & Employee Management Module \\
		                                                      & Services Module            \\
		                                                      & Shop Module                \\
		\midrule

		\multirow{1}{0.3\textwidth}{Software Decision Module} & Database Driver Module     \\

		\bottomrule
	\end{tabular}
	\caption{Module Hierarchy}
	\label{TblMH}
\end{table}

\section{Connection Between Requirements and Design} \label{SecConnection}

The design of the system is intended to satisfy the requirements developed in the SRS. In this
stage, the system is decomposed into modules. The connection between requirements and modules is
listed in Table \ref{TblRT}.

\section{Module Decomposition} \label{SecMD}

Modules are decomposed according to the principle of ``information hiding'' proposed by
\citet{ParnasEtAl1984}. The \emph{Secrets} field in a module decomposition is a brief statement of
the design decision hidden by the module. The \emph{Services} field specifies \emph{what} the
module will do without documenting \emph{how} to do it. For each module, a suggestion for the
implementing software is given under the \emph{Implemented By} title. If the entry is \emph{OS},
this means that the module is provided by the operating system or by standard programming language
libraries. \emph{\progname{}} means the module will be implemented by the \progname{} software.

Only the leaf modules in the hierarchy have to be implemented. If a dash (\emph{--}) is shown, this
means that the module is not a leaf and will not have to be implemented.

\subsection{Hardware Hiding Modules (\ref{mHH})}

\begin{description}
	\item[Secrets:]The data structure and algorithm used to implement the virtual hardware.
	\item[Services:]Serves as a virtual hardware used by the rest of the system. This module provides the
	interface between the hardware and the software. So, the system can use it to display outputs or to
	accept inputs.
	\item[Implemented By:] OS
\end{description}

\subsection{Behaviour-Hiding Module}

% \begin{description}
%      \item[Secrets:] The contents of the required behaviours.
%      \item[Services:] Includes programs that provide externally visible behaviour of the system as specified
%              in the software requirements specification (SRS) documents. This module serves as a communication
%              layer between the hardware-hiding module and the software decision module. The programs in this
%              module will need to change if there are changes in the SRS.
%      \item[Implemented By:] --
% \end{description}

% \subsubsection{Input Format Module (\mref{mInput})}

% \begin{description}
%      \item[Secrets:] The format and structure of the input data.
%      \item[Services:] Converts the input data into the data structure used by the input parameters module.
%      \item[Implemented By:] [Your Program Name Here]
%  \item[Type of Module:] [Record, Library, Abstract Object, or Abstract Data Type] [Information to include
%              for leaf modules in the decomposition by secrets tree.]
% \end{description}

\subsubsection{User Module (\ref{mUsers})}

\begin{description}
	\item[Secrets:] The format and structure of user data objects.
	\item[Services:] Provide CRUD operations for user objects, and the ability to verify a user's
		authorization.
	\item[Implemented By:] \progname{}
	\item[Type of Module:] Library
\end{description}

\subsubsection{Quotes Module (\ref{mQuotes})}

\begin{description}
	\item[Secrets:] The format and structure of quote and chat message data objects.
	\item[Services:] Provide CRUD operations for quote and chat message objects.
	\item[Implemented By:] \progname{}
	\item[Type of Module:] Library
\end{description}

\subsubsection{Appointments Module (\ref{mAppointments})}

\begin{description}
	\item[Secrets:] The format and structure of appointment data objects.
	\item[Services:] Provide CRUD operations for appointment objects, the ability to verify if an appointment
		is acceptable (i.e., is possible to fulfill), and the ability to accept an appointment and reject
		all conflicting appointments.
	\item[Implemented By:] \progname{}
	\item[Type of Module:] Library
\end{description}

\subsubsection{Work Orders Module (\ref{mWorkOrders})}

\begin{description}
	\item[Secrets:] The format and structure of work order data objects.
	\item[Services:] Provide CRUD operations for work order objects.
	\item[Implemented By:] \progname{}
	\item[Type of Module:] Library
\end{description}

\subsubsection{Employee Management Module (\ref{mEmployeeManagement})}

\begin{description}
	\item[Secrets:] --
	\item[Services:] Provide CRUD operations for managing employee's relation with shop.
	\item[Implemented By:] \progname{}
	\item[Type of Module:] Library
\end{description}

\subsubsection{Services Module (\ref{mServices})}

\begin{description}
	\item[Secrets:] The format and structure of service data objects.
	\item[Services:] Provide CRUD operations for service objects.
	\item[Implemented By:] \progname{}
	\item[Type of Module:] Library
\end{description}

\subsubsection{Shop Module (\ref{mShop})}

\begin{description}
	\item[Secrets:] The format and structure of shop data objects.
	\item[Services:] Provides CRUD operations for shop objects and the ability to query for a shop by
		distance, postal code and/or shop name.
	\item[Implemented By:] \progname{}
	\item[Type of Module:] Library
\end{description}

\subsection{Software Decision Module}

% \begin{description}
%      \item[Secrets:] The design decision based on mathematical theorems, physical facts, or programming
%              considerations. The secrets of this module are \emph{not} described in the SRS.
%      \item[Services:] Includes data structure and algorithms used in the system that do not provide direct
%              interaction with the user.
%              % Changes in these modules are more likely to be motivated by a desire to
%              % improve performance than by externally imposed changes.
%      \item[Implemented By:] --
% \end{description}

\subsubsection{Database Driver Module (\ref{mDBDriver})}
\begin{description}
	\item[Secrets:] The data structures and algorithms for representing the database's schema and
		communicating with the database.
	\item[Services:] Provides a driver to interface with the database
	\item[Implemented By:] \progname{}, Prisma
\end{description}

\newpage

\section{Traceability Matrix} \label{SecTM}

This section shows two traceability matrices: between the modules and the requirements and between
the modules and the anticipated changes.

\begin{longtable}{p{0.2\textwidth} p{0.6\textwidth}}
	\caption{Trace Between Requirements and Modules} \label{TblRT}                                                                                 \\
	\toprule
	\textbf{Requirements} & \textbf{Modules}                                                                                                       \\
	\midrule
	BE1                   & \ref{mUsers}                                                                                                           \\
	BE2                   & \ref{mUsers}                                                                                                           \\
	BE3                   & \ref{mUsers}                                                                                                           \\
	BE4                   & \ref{mAppointments}                                                                                                    \\
	BE5                   & \ref{mAppointments}                                                                                                    \\
	BE6                   & \ref{mAppointments}                                                                                                    \\
	BE7                   & \ref{mAppointments}                                                                                                    \\
	BE8                   & \ref{mQuotes}                                                                                                          \\
	BE9                   & \ref{mQuotes}                                                                                                          \\
	BE10                  & \ref{mQuotes}                                                                                                          \\
	BE11                  & \ref{mQuotes}                                                                                                          \\
	BE12                  & \ref{mQuotes}                                                                                                          \\
	BE13                  & \ref{mQuotes}                                                                                                          \\
	BE14                  & \ref{mQuotes}                                                                                                          \\
	BE15                  & \ref{mQuotes}                                                                                                          \\
	BE16                  & \ref{mQuotes}                                                                                                          \\
	BE17                  & \ref{mWorkOrders}                                                                                                      \\
	BE18                  & \ref{mWorkOrders}                                                                                                      \\
	BE19                  & \ref{mWorkOrders}                                                                                                      \\
	BE20                  & \ref{mWorkOrders}                                                                                                      \\
	BE21                  & \ref{mWorkOrders}                                                                                                      \\
	BE22                  & \ref{mWorkOrders}                                                                                                      \\
	BE23                  & \ref{mWorkOrders}                                                                                                      \\
	BE24                  & \ref{mEmployeeManagement}                                                                                              \\
	BE25                  & \ref{mEmployeeManagement}                                                                                              \\
	BE26                  & \ref{mEmployeeManagement}                                                                                              \\
	BE27                  & \ref{mEmployeeManagement}                                                                                              \\
	BE28                  & \ref{mServices}                                                                                                        \\
	BE29                  & \ref{mServices}                                                                                                        \\
	BE30                  & \ref{mServices}                                                                                                        \\
	BE31                  & \ref{mServices}                                                                                                        \\
	BE32                  & \ref{mShop}                                                                                                            \\
	BE33                  & \ref{mShop}                                                                                                            \\
	LF1                   & \ref{mUsers} \ref{mQuotes} \ref{mAppointments} \ref{mWorkOrders} \ref{mEmployeeManagement} \ref{mServices} \ref{mShop} \\
	LF2                   & \ref{mUsers} \ref{mQuotes} \ref{mAppointments} \ref{mWorkOrders} \ref{mEmployeeManagement} \ref{mServices} \ref{mShop} \\
	LF3                   & \ref{mUsers} \ref{mQuotes} \ref{mAppointments} \ref{mWorkOrders} \ref{mEmployeeManagement} \ref{mServices} \ref{mShop} \\
	UH1                   & \ref{mUsers} \ref{mQuotes} \ref{mAppointments} \ref{mWorkOrders} \ref{mEmployeeManagement} \ref{mServices} \ref{mShop} \\
	UH2                   & \ref{mUsers} \ref{mQuotes} \ref{mAppointments} \ref{mWorkOrders} \ref{mEmployeeManagement} \ref{mServices} \ref{mShop} \\
	UH3                   & \ref{mUsers} \ref{mQuotes} \ref{mAppointments} \ref{mWorkOrders} \ref{mEmployeeManagement} \ref{mServices} \ref{mShop} \\
	PR1                   & \ref{mUsers} \ref{mQuotes} \ref{mAppointments} \ref{mWorkOrders} \ref{mEmployeeManagement} \ref{mServices} \ref{mShop} \\
	OE1                   & \ref{mUsers} \ref{mQuotes} \ref{mAppointments} \ref{mWorkOrders} \ref{mEmployeeManagement} \ref{mServices} \ref{mShop} \\
	MS1                   & \ref{mUsers} \ref{mQuotes} \ref{mAppointments} \ref{mWorkOrders} \ref{mEmployeeManagement} \ref{mServices} \ref{mShop} \\
	SR1                   & \ref{mUsers} \ref{mQuotes} \ref{mAppointments} \ref{mWorkOrders} \ref{mEmployeeManagement} \ref{mServices} \ref{mShop} \\
	SR2                   & \ref{mUsers} \ref{mQuotes} \ref{mAppointments} \ref{mWorkOrders} \ref{mEmployeeManagement} \ref{mServices} \ref{mShop} \\
	CR1                   & \ref{mUsers} \ref{mQuotes} \ref{mAppointments} \ref{mWorkOrders} \ref{mEmployeeManagement} \ref{mServices} \ref{mShop} \\
	CR2                   & \ref{mUsers} \ref{mQuotes} \ref{mAppointments} \ref{mWorkOrders} \ref{mEmployeeManagement} \ref{mServices} \ref{mShop} \\
	LR1                   & \ref{mUsers} \ref{mQuotes} \ref{mAppointments} \ref{mWorkOrders} \ref{mEmployeeManagement} \ref{mServices} \ref{mShop} \\
	\bottomrule
\end{longtable}

\begin{table}[H]
	\caption{Trace Between Anticipated Changes and Modules} \label{TblACT}
	\centering
	\begin{tabular}{p{0.2\textwidth} p{0.6\textwidth}}
		\toprule
		\textbf{AC}                 & \textbf{Modules}                                                                                                                       \\
		\midrule
		\ref{acHardware}            & \ref{mHH}                                                                                                                              \\
		\ref{acInitialInput}        & \ref{mDBDriver} \ref{mUsers} \ref{mQuotes} \ref{mAppointments} \ref{mWorkOrders} \ref{mEmployeeManagement} \ref{mServices} \ref{mShop} \\
		\ref{acServerSideResponses} & \ref{mDBDriver} \ref{mUsers} \ref{mQuotes} \ref{mAppointments} \ref{mWorkOrders} \ref{mEmployeeManagement} \ref{mServices} \ref{mShop} \\
		\ref{acStyling}             & \ref{mUsers} \ref{mQuotes} \ref{mAppointments} \ref{mWorkOrders} \ref{mEmployeeManagement} \ref{mServices} \ref{mShop}                 \\
		\ref{acFormatting}          & \ref{mUsers} \ref{mQuotes} \ref{mAppointments} \ref{mWorkOrders} \ref{mEmployeeManagement} \ref{mServices} \ref{mShop}                 \\
		\bottomrule
	\end{tabular}
\end{table}

\section{Use Hierarchy Between Modules} \label{SecUse}

In this section, the uses hierarchy between modules is provided. \citet{Parnas1978} said of two
programs A and B that A {\em uses} B if correct execution of B may be necessary for A to complete
the task described in its specification. That is, A {\em uses} B if there exist situations in which
the correct functioning of A depends upon the availability of a correct implementation of B. Figure
\ref{FigUH} illustrates the use relation between the modules. It can be seen that the graph is a
directed acyclic graph (DAG). Each level of the hierarchy offers a testable and usable subset of
the system, and modules in the higher level of the hierarchy are essentially simpler because they
use modules from the lower levels.

\begin{figure}[H]
	\centering
	\includegraphics[width=0.7\textwidth]{UsesHierarchy.png}
	\caption{Use hierarchy among modules}
	\label{FigUH}
\end{figure}

\newpage

\bibliographystyle{plainnat}
\bibliography{../../../refs/References}

\newpage

\section{Appendix}

\end{document}
