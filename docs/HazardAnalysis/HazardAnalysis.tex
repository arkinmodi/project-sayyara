\documentclass{article}

\usepackage{booktabs}
\usepackage{tabularx}
\usepackage{hyperref}
\usepackage{pdflscape}
\usepackage{float}
\usepackage{adjustbox}
\usepackage{multirow}
\usepackage{enumitem}
\usepackage{longtable}

\hypersetup{
	colorlinks=true,		% false: boxed links; true: colored links
	linkcolor=red,			% color of internal links (change box color with linkbordercolor)
	citecolor=green,		% color of links to bibliography
	filecolor=magenta,	% color of file links
	urlcolor=cyan				% color of external links
}

\title{Hazard Analysis\\\progname}

\author{\authname}

\date{\today}

\input{../Comments}
%% Common Parts

\newcommand{\progname}{Software Engineering}
\newcommand{\authname}{Team 3, Tiny Coders
	\\ Arkin Modi
	\\ Joy Xiao
	\\ Leon So
	\\ Timothy Choy} % AUTHOR NAMES

\usepackage{hyperref}
\hypersetup{colorlinks=true, linkcolor=blue, citecolor=blue, filecolor=blue,
	urlcolor=blue, unicode=false}
\urlstyle{same}

\usepackage{parskip}
\usepackage{geometry}
\geometry{a4paper, portrait, margin=1in}


\begin{document}

\maketitle
\thispagestyle{empty}

~\newpage

\pagenumbering{roman}

\begin{table}[hp]
	\caption{Revision History} \label{TblRevisionHistory}
	\begin{tabularx}{\textwidth}{llX}
		\toprule
		\textbf{Date}    & \textbf{Developer(s)} & \textbf{Change}                                                                \\
		\midrule
		October 13, 2022 & Arkin Modi            & Create Failure Mode and Effect Analysis table                                  \\
		October 14, 2022 & Joy Xiao              & Introduction                                                                   \\
		October 16, 2022 & Arkin Modi            & Fill in FMEA table for Work Orders, Shop Profile, Services, and Shop Employees \\
		\bottomrule
	\end{tabularx}
\end{table}

~\newpage

\tableofcontents

~\newpage

\pagenumbering{arabic}

\section{Introduction}
This document outlines the hazard analysis of Sayyara. The definition of hazard is any property or
condition in the system along with conditions in the environment that may cause harm or damage.
This definition is from Nancy Leveson's work. The hazards for Sayyara include security and usage
hazards such as protecting personal information, database failures, and having no internet
connection.

\section{Scope and Purpose of Hazard Analysis}
The scope of the hazard analysis is to identify any hazards that may arise when using the
application, their causes, coming up with steps to eliminate or mitigate the effect of the hazard.
The purpose of the hazard analysis is to pinpoint areas where hazards may arise and their effects
and come up with mitigation steps. Through completing the hazard analysis, safety and security
requirements will be developed early in the design process to minimize the risk of having hazards
occur without plans in place to reduce or mitigate the effects.

\section{System Boundaries and Components}
The system consists of:
\begin{enumerate}
	\item The application's front-end and back-end components in the major categories:
	      \begin{itemize}
		      \item Authentication
		      \item Appointments
		      \item Quotes
		      \item Work Orders
		      \item Shop Profile
		      \item Services
		      \item Shop Employees
	      \end{itemize}
	\item The database being used which will store all of application's data
\end{enumerate}

\section{Critical Assumptions}
\begin{itemize}
	\item Assume the users of the application are not intentionally trying to misuse it
	\item Assume that the user will follow cybersecurity best practices
\end{itemize}

\newpage
\begin{landscape}
	\section{Failure Mode and Effect Analysis}
	\begin{longtable}{|p{0.15\textwidth}|p{0.15\textwidth}|p{0.25\textwidth}|p{0.25\textwidth}|p{0.35\textwidth}|p{0.1\textwidth}|p{0.05\textwidth}|}
		\caption{Failure Mode and Effect Analysis Table}                                                                \\
		\hline
		\multicolumn{1}{|c|}{\textbf{Component}}
		 & \multicolumn{1}{|c|}{\textbf{Failure Modes}}
		 & \multicolumn{1}{|c|}{\textbf{Effects of Failure}}
		 & \multicolumn{1}{|c|}{\textbf{Causes of Failure}}
		 & \multicolumn{1}{|c|}{\textbf{Recommended Action}}
		 & \multicolumn{1}{|c|}{\textbf{SR}}
		 & \multicolumn{1}{|c|}{\textbf{Ref.}}                                                                          \\
		\hline
		\multirow{2}{*}{Work Orders}
		 & Work Order is missing
		 & Customer and Employees will not know any of the work that has been done for a specific job
		 & \begin{enumerate}[label=\alph*., leftmargin=*]
			   \item Database failure
		   \end{enumerate}
		 & \begin{enumerate}[label=\alph*., leftmargin=*]
			   \item Regular and automatic database backups/snapshots and allow shop owners to request rollbacks
		   \end{enumerate}
		 & \begin{enumerate}[label=\alph*., leftmargin=*]
			   \item
		   \end{enumerate}
		 & H5-1                                                                                                         \\
		\cline{2-7}
		~
		 & Work Order is missing detailed information
		 & Customer and Employees will not know all of the work that has been done for a specific job
		 & \begin{enumerate}[label=\alph*., leftmargin=*]
			   \item Database failure
		   \end{enumerate}
		 & \begin{enumerate}[label=\alph*., leftmargin=*]
			   \item Refer to H5-1a
		   \end{enumerate}
		 & \begin{enumerate}[label=\alph*., leftmargin=*]
			   \item
		   \end{enumerate}
		 & H5-2                                                                                                         \\
		\hline
		Shop Profile
		 & Unable to find details of a shop
		 & Customers will not be able to see the information about a specific shop (e.g., address, phone number, etc.)
		 & \begin{enumerate}[label=\alph*., leftmargin=*]
			   \item Database failure
		   \end{enumerate}
		 & \begin{enumerate}[label=\alph*., leftmargin=*]
			   \item Refer to H5-1a
		   \end{enumerate}
		 & \begin{enumerate}[label=\alph*., leftmargin=*]
			   \item
		   \end{enumerate}
		 & H6-1                                                                                                         \\
		\hline
		\multirow{2}{*}{Services}
		 & Unable to find a service
		 & Customers and Employees will not be able to see what services are offered by the shop
		 & \begin{enumerate}[label=\alph*., leftmargin=*]
			   \item Database failure
			   \item Client-side network failure
		   \end{enumerate}
		 & \begin{enumerate}[label=\alph*., leftmargin=*]
			   \item Refer to H5-1a
			   \item Display a message to the user informing them that they are offline
		   \end{enumerate}
		 & \begin{enumerate}[label=\alph*., leftmargin=*]
			   \item
		   \end{enumerate}
		 & H7-1                                                                                                         \\
		\cline{2-7}
		~
		 & Unable to find details of a service
		 & Customers and Employees will not be able to see the details of a service (e.g., price, estimated time, etc.)
		 & \begin{enumerate}[label=\alph*., leftmargin=*]
			   \item Database failure
			   \item Client-side network failure
		   \end{enumerate}
		 & \begin{enumerate}[label=\alph*., leftmargin=*]
			   \item Refer to H5-1a
			   \item Refer to H7-1b
		   \end{enumerate}
		 & \begin{enumerate}[label=\alph*., leftmargin=*]
			   \item
		   \end{enumerate}
		 & H7-2                                                                                                         \\
		\hline
		\multirow{2}{*}{Shop Employees}
		 & A former employee joins the shop account
		 & The former employee can view sensitive information and perform unauthorized actions
		 & \begin{enumerate}[label=\alph*., leftmargin=*]
			   \item A former employee accepts their invite link to join the shop as an employee after their employment
			         has been terminated
		   \end{enumerate}
		 & \begin{enumerate}[label=\alph*., leftmargin=*]
			   \item Invite links should expire after a set period of time
			   \item Invite links should only be able to be accepts once
			   \item Shop owners should be able to revoke access to any employee
		   \end{enumerate}
		 & \begin{enumerate}[label=\alph*., leftmargin=*]
			   \item
		   \end{enumerate}
		 & H8-1                                                                                                         \\
		\cline{2-7}
		~
		 & An employee of the shop cannot be found
		 & Shop owners will not be able view employee's details; Effected employee will not be able to be
		assigned to work orders or view their profile details
		 & \begin{enumerate}[label=\alph*., leftmargin=*]
			   \item Database failure
			   \item Client-side network failure
		   \end{enumerate}
		 & \begin{enumerate}[label=\alph*., leftmargin=*]
			   \item Refer to H5-1a
			   \item Refer to H7-1b
			   \item Employees should be able to be reinvited to a shop
		   \end{enumerate}
		 & \begin{enumerate}[label=\alph*., leftmargin=*]
			   \item
		   \end{enumerate}
		 & H8-2                                                                                                         \\
		\hline
	\end{longtable}
\end{landscape}

\section{Safety and Security Requirements}

\wss{Newly discovered requirements.  These should also be added to the SRS.  (A
	rationale design process how and why to fake it.)}

\section{Roadmap}

\wss{Which safety requirements will be implemented as part of the capstone timeline?
	Which requirements will be implemented in the future?}

\wss{The Roadmap is where you can explain which safety requirements will be implemented during the
	course, and which will be ``postponed'' until after the course.}

\end{document}
