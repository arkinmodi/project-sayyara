\documentclass{article}

\usepackage{booktabs}
\usepackage{tabularx}
\usepackage{hyperref}

\hypersetup{
    colorlinks=true,       % false: boxed links; true: colored links
    linkcolor=red,          % color of internal links (change box color with linkbordercolor)
    citecolor=green,        % color of links to bibliography
    filecolor=magenta,      % color of file links
    urlcolor=cyan           % color of external links
}

\title{Hazard Analysis\\\progname}

\author{\authname}

\date{}

\input{../Comments}
%% Common Parts

\newcommand{\progname}{Software Engineering}
\newcommand{\authname}{Team 3, Tiny Coders
	\\ Arkin Modi
	\\ Joy Xiao
	\\ Leon So
	\\ Timothy Choy} % AUTHOR NAMES

\usepackage{hyperref}
\hypersetup{colorlinks=true, linkcolor=blue, citecolor=blue, filecolor=blue,
	urlcolor=blue, unicode=false}
\urlstyle{same}

\usepackage{parskip}
\usepackage{geometry}
\geometry{a4paper, portrait, margin=1in}


\begin{document}

\maketitle
\thispagestyle{empty}

~\newpage

\pagenumbering{roman}

\begin{table}[hp]
	\caption{Revision History} \label{TblRevisionHistory}
	\begin{tabularx}{\textwidth}{llX}
		\toprule
		\textbf{Date} & \textbf{Developer(s)} & \textbf{Change}        \\
		\midrule
		October 13    & Joy Xiao              & Introduction           \\
		Date2         & Name(s)               & Description of changes \\
		...           & ...                   & ...                    \\
		\bottomrule
	\end{tabularx}
\end{table}

~\newpage

\tableofcontents

~\newpage

\pagenumbering{arabic}

\wss{You are free to modify this template.}

\section{Introduction}
This document outlines the hazard analysis of Sayyara. The definition of hazard is any property or
condition in the system along with conditions in the environment that may cause harm or damage.
This definition is from Nancy Leveson's work. The hazards for Sayyara include security and usage
hazards such as protecting personal information, database failures, and no internet connection.

\section{Scope and Purpose of Hazard Analysis}
The scope of the hazard analysis is to identify any hazards that may arise when using the
application and coming up with mitigation steps. The purpose of the hazard analysis is to pinpoint
areas where hazards may arise and their effects and come up with mitigation steps. Through
completing the hazard analysis, safety and security requirements will be developed early in the
design process to prevent hazards from occurring without plans in place to reduce the risk.

\section{System Boundaries and Components}
The system consists of: \list{enumerate}
\item The application's front-end and back-end components
\item The database being used
\end{enumerate}

\section{Critical Assumptions}

\wss{These assumptions that are made about the software or system.  You should
	minimize the number of assumptions that remove potential hazards.  For instance,
	you could assume a part will never fail, but it is generally better to include
	this potential failure mode.}

There are no critical assumptions made.

\section{Failure Mode and Effect Analysis}

\wss{Include your FMEA table here}

\section{Safety and Security Requirements}

\wss{Newly discovered requirements.  These should also be added to the SRS.  (A
	rationale design process how and why to fake it.)}

\section{Roadmap}

\wss{Which safety requirements will be implemented as part of the capstone timeline?
	Which requirements will be implemented in the future?}

\end{document}
